\documentclass{article}

\usepackage{url}

\begin{document}

\title{Personal Statement}
\author{Zening Qu}
\maketitle

\section{Design of the system}


One of the most interesting problems I can see within MB is this: How should multimedia data be organized/presented/manipulated in a 3D virtual space? We are in an age where storing multimedia data has been really cheap and easy: all you need is a disk and perhaps a database. Presenting multimedia in a 2D space is quite common practice too: we can create albums, tag, sort, favorite, comment, and share these data. But would these operations, which are validated and quite popular in the 2D world, still make sense in the 3D space? Would new ways of interacting with these data emerge? In the 2D world, it's all about windows, buttons, mouse clicks and key strokes. These classic ways of interaction may or may be proper for 3D spaces. As an example, what if we make photos stack together as if they were on a postcard stack (see figure 1)? Users can still take a closer look at each of the photos, but not necessarily via a pop up window -- we can just zoom the camera in so that it feels like that the user is bending over to a postcard and trying to see some more details...

Text messages could behave like colorful little tags (see figure 2) that are hang on thin strings just above the users' eyes. These will look like chains of colorful LED lights when seen from faraway places. When you get closer to them you can "grab" one and examine its content. Once again, not necessarily in a window.

I am trying to read more in artificial intelligence \& machine learning lately, as I see it as a new trend in designing intellectual interactive systems. I might be able to come up with some more ideas to make the 3D space behave more "smart" somehow, but I don't know for now!

Apart from these, many design rules and storytelling principles applies to the design of the system. And I believe you and Fabian know the best! :D I am just suggesting that we should have a clear set of rules and goals in the design process of the system. We then apply these rules to CQM, and evaluate the effectiveness of the system (possibly using a case study).

I am going to try to write about the evaluation process shortly but before that, I'd like to give a few example projects:
CityFlocks: Designing social navigation for urban mobile information systems. Proceedings of the ACM Conference on Designing Interactive Systems. \url{http://eprints.qut.edu.au/10871/1/dis2008_v13mf_mb.pdf}

Photos on the go: A mobile application case study. Proceedings of the ACM Conference on Human Factors in Computing Systems. \url{http://infolab.stanford.edu/~mor/research/Naaman-chi08-photosmobile.pdf}

Both projects describe systems designed with a specific goal in mind, filling an existing gap. The evaluation methodologies such as surveys, interviews and diaries can be used by us too. These projects are not alone, there are many more similar ones :)

As for evaluation, these are the methodologies that I've thought about:

Surveys - good for gathering massive, shallow data. We can get a rough idea about how well the system is designed by asking Likert Scale questions such as :"In a scale of 10 how comfortable would you rate the system's zoom in and zoom out process?" There are survey templates which are well established and validated that we can use in our usability test. A collection of templates here.
Lastly, surveys are cheap, which is a good thing but also often make them the most abused (and criticized).

System Logs - we could program CQM so that user operations are logged automatically (How many times does the user use a certain feature? How much time does a user spend in using a feature? How much time is spent in switching between features? Are there any features unused? ...). This will give us quite good raw data about the interface efficiency \& effectiveness as long as the users are not distracted by other things while navigating in the world (which means someone should probably by standing there and getting rid of potential sources of distractions).

Observation (Interview) - videotape while the user is interacting with the system. The tape could later be coded/annotated and analyzed using qualitative analysis methods. This could be followed up by interview questions asking in-depth questions about the system. This is, expensive, unfortunately, but not totally undoable. Let's say that one day in the future a group of pupils want to visit the system in Chiparaki, then it would be a great chance for us to do such interviews with them. 


\section{Understanding the users}


Another direction in which we can think and write is concerned with the users instead of the system. This system trys to present to a community its own history and culture. What is the user expectation of that? What is the user preference/habits in such situations? This might sound like the first direction at first, but I think understanding users is more about understanding the nature of the problem, which could guide the design. For example, if we understand the nature of the problem, we would have a reason for organizing photos in certain ways (a postcard stand, a photo collage, or other forms). For example, we would be able to state that organizing photos on a postcard stand is a wise choice because it agrees with the users' expectation (and they back up our statement with experiment data).

An example study of similar nature is this: Malone, T.W. (1983) How do people organize their desks? Implications for the design of office information systems. ACM Transactions on Information Systems. I admire the paper a lot because I believe it gave the reason why applications and files should be organized in a desktop system in the way they are organized today.

Besides finding the expectations/habits of groups of users, we can also look for inter-group difference. I suppose Buenos Aires commuters would have different habits than downtown LA pupils? A study of similar nature is this: Children's interests adn concerns when using the International Children's Digital Library: A four-country case study. Proceedings of the ACM/IEE-CS Joint Conference on Digital Libraries.

While it's nice and fun to think about, I think this direction is going a little too far from our current position... It's looking more like a social/behavior science study rather than a system design :P

\end{document}